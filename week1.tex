\documentclass[a4paper,parskip=half]{scrartcl}
\usepackage[english]{babel}
\usepackage[utf8]{inputenc}
\usepackage[T1]{fontenc}
\usepackage{mathptmx}
\usepackage{amsmath}
\usepackage{amssymb}
\usepackage{bm}
\usepackage{clrscode3e}
\usepackage[headsepline]{scrpage2}
\usepackage{enumitem}

\thispagestyle{scrplain}
\setlength{\headsep}{0.5em}
\setlength{\parskip}{0.3em}

% add \Infty symbol

\DeclareSymbolFont{Symbols}{OMS}{zplm}{m}{n}% Palatino
\DeclareMathSymbol{\Infty}{\mathord}{Symbols}{"31}

% solution environment macro

\newenvironment{solution}[1]{
%\begin{minipage}{\textwidth}
\rule{\textwidth}{1pt}
\begin{description}[leftmargin=3em, style=nextline, topsep=0em,
font={\bfseries\rmfamily}]
\item[#1]
}{
\end{description}
%\end{minipage}
}

% header

\ihead[\bf{Design and Analysis of Algorithms, Fall 2014\\Exercise II:
Solutions}]{} \ohead[Jarno Leppänen\\014340691]{}

% begin document

\begin{document}

\begin{solution}{II-1}

Define matrix operator $(\star)$ as follows: $C = A \star B \iff c_{ij} =
\min_{k} \left\{ a_{ik} + b_{kj} \right\}$. Now the shortest paths between all
pairs of vertices can be written as $\Delta = L^{(n-1)} = L^{(n-2)} \star W =
(L^{(n-3)} \star W) \star W = \underbrace{ ( \ldots (L^{(1)} \star W) \star
\ldots \star W) \star W }_{n-1} = W^{n-1}$, where $\delta_{ij}$ is the shortest
path between vertices $i$ and $j$, $l_{ij}^{(m)}$ is the shortest path between
vertices $i$ and $j$ consisting of at most $m$ edges and $w_{ij}$ the length of
the edge between vertices $i$ and $j$. The exponentiation in $W^{n-1}$ means
that the operator $(\star)$ is applied $n-1$ times from left to right.

The procedure for \proc{Slow-All-Pairs-Shortes-Paths} applies $(\star)$ in this
order, but \proc{Faster-All-Pairs-Shortest-Paths} makes use of the presumed
associativity of $(\star)$, that is, $A \star (B \star C) = (A \star B) \star
C$, so that the computation can be reordered as $W^{2^{\lceil \lg(m-1) \rceil}}
= \left( W^{2^{\lceil \lg(m-1 \rceil - 1}} \right) \star \left( W^{2^{\lceil
\lg(m-1) \rceil - 1}} \right) = \ldots$ and only $\lceil \lg(n) \rceil$
applications of $(\star)$ need to be performed.

\end{solution}

\begin{solution}{II-2}

A monotonically increasing subsequence of a sequence $X$ is also a subsequence
of $S$ which consists of elements of $X$ sorted in ascending order (proof
omitted). Therefore the following algorithm for finding the longest
monotonically increasing subsequence can be constructed:

\begin{codebox}
\Procname{$\proc{Longest-Monotonically-Increasing-Subsequence}(X)$}
\li $S =
\proc{Merge-Sort}(X)$
\li \Return $\proc{LCS-Length}(X,S)$
\end{codebox}

The running time of \proc{Merge-Sort} is $\Theta(n \lg n)$, where $n$ is the
length of sequence $X$. The running time of \proc{LCS-Length} is $\Theta(n^2)$
(CLRS 15.4) which makes the overall running time of
\proc{Longest-Monotonically-Increasing-Subsequence} $\Theta(n^2)$.

\end{solution}

\begin{solution}{II-3}

Label the points in the set from $p_1$ to $p_n$ by their x-coordinate so that
point $p_1$ is the leftmost point and $p_n$ is the rightmost.

A bitonic path $P_{ij}, i \leq j$ includes all points from 1 to $j$ and
consists of two parts: a leftgoing path from $p_i$ to $p_1$ and a rightgoing
path from $p_1$ to $p_j$. By definition, the rightmost point $p_j$ is always on
the rightgoing path.

Let $b(i, j)$ be the shortest bitonic path $P_{ij}$ and $l(i, j)$ the euclidian
distance between points $p_i$ and $p_j$. The shortest bitonic path for two
points is simply the length of the path and therefore $b(1,2) = l(1,2)$.  Also,
for $j > i+1$, $b(i, j) = b(i, j-1) + l(j-1,j)$ because point $p_{j-1}$ is part
of the rightgoing path. If $p_{j-1}$ is on the leftgoing part, $p_j$ has a
predecessor $p_k, k < j-1$ on the rightgoing path and the subpath from $p_k$ to
$p_{j-1}$ must be the shortest bitonic path $b(k,j-1)$. Therefore $b(j-1,j) =
\min_{k<j-1} \left\{ b(k,j-1) + l(k,j) \right\}$. For the degenerate case $b(j,
j) = b(j-1,j) + l(j-1,j)$, because on the bitonic tour one of the adjacent
points to $p_n$ must be $p_{n-1}$. From these recurrences the following
algorithm can be constructed:

\begin{codebox}
\Procname{$\proc{Shortest-Bitonic-Tour}(X)$}
\li $P = \proc{Merge-Sort}(X)$
\li $b(1,2) = l(1,2)$
\li \For $j = 3$ \To $n$ \Do
\li   \For $i = 1$ \To $j-2$ \Do
\li     $b(i,j) = b(i,j-1) + l(j-1,j)$ \End
\li   $b(j-1,j) = \Infty$
\li   \For $k = 1$ \To $j-2$ \Do
\li     $b(j-1,j) = \min(b(j-1,j), b(k,j-1) + l(k,j))$ \End \End
\li $b(n,n) = b(n-1,n)+l(n-1,n)$
\li \Return $b$
\end{codebox}

The running time of \proc{Merge-Sort} is $O(\lg n)$. The outer loop in the
procedure iterates $n-2$ times and the inner loops $n-2$ times at most which
gives a total running time of $O(n^2)$ for \proc{Shortest-Bitonic-Tour}.

\end{solution}

\begin{solution}{II-4}

Let $c(i,j) = M - j + i - \sum_{k=i}^j l_k$ be the extra characters left out
when laying out words $i$ through $j$ on a line. Now define the penalty
associated with this line as follows:
\begin{aligned}
p(i,j) =
\begin{cases}
\Infty & \text{if } c(i,j) < 0,\\
0 & \text{if } j = n \text{ and } c(i,j) \geq 0,\\
c(i, j)^3 & \text{otherwise}.
\end{cases}
\end{aligned}

Note that if $c(i,j) < 0$, the words do not fit in the line.

Let $b(j)$ be the total cost of an optimal layout of words $1$ through $j$.  If
the last line begins with word $i \leq j$, the optimal cost would be $b(j) =
b(i-1) + l(i,j)$. The cost of words from $1$ to $i$ must be optimal because
otherwise the last line would not begin with $i$th word. We can therefore try
every $i$ and define the total cost recursively as $b(j) = \min_{i \leq j}
\left\{ b(i-1) + p(i,j) \right\}, j > 0$ and $b(0) = 0$.

The extra characters $c(i,j)$ and line penalties $p(i,j)$ can be precalculated.
The breaking first words $w(k)$ on each line $k$  must also be tracked while
calculating optimal costs $b(j)$. The algorithm can now be given as follows:

\begin{codebox}
\Procname{$\proc{Optimal-Layout}(L)$}
\li \For $i = 1$ \To $n$ \Do
\li   $c(i,i) = M - l_i$
\li   \For $j = i + 1$ \To $n$ \Do
\li     $c(i,j) = c(i,j-1) - l_j - 1$ \End \End
\li \For $i = 1 \To n$ \Do
\li   \For $j = 1 \To n$ \Do
\li     \If $c(i,j) < 0$ \Then
\li       $p(i,j) = \Infty$
\li     \ElseIf $j = n$ \textbf{and} $c(i,j) \geq 0$ \Then
\li       $p(i,j) = 0$
\li     \Else
\li       $p(i,j) = c(i,j)^3$ \End \End \End
\li $b(0) = 0$
\li \For $j = 1 \To n$ \Do
\li   $b(j) = \Infty$
\li   \For $i = 1 \To j$ \Do
\li     \If $b(j) > b(i-1) + p(i,j)$ \Then
\li       $b(j) = b(j-1) + p(i,j)$
\li       $w(j) = i$ \End \End \End
\li \Return $b(n), W$
\end{codebox}

The running time of each loop and hence the total procedure
\proc{Optimal-Layout} are $\Theta(n^2)$. The space complexity is also
$\Theta(n^2)$.

All lines can be printed with the following procedure given $W$ and
called with the total number of words $n$:

\begin{codebox}
\Procname{$\proc{Print-Layout}(W,j)$}
\li $i = p(j)$
\li \If $i = 1$ \Then
\li   $k = 1$
\li \Else
\li   $k = \proc{Print-Layout}(W,i-1) + 1$ \End
\li \For $m = i \To j$ \Do
\li   \textbf{print} $m$ \End
\li \textbf{println}
\li \Return $k$
\end{codebox}

Here \textbf{print} $m$ prints the word $m$ and \textbf{println} prints a
line break. \

\end{solution}

\begin{solution}{II-5}

If at any point during the procedure we encounter a negative weight on the
diagonal of $L^{(m)}$ i.e. $l_{ii}^{(m)} < 0$ for any $i$, there exists a negative cycle
containing $i$ that has at most $m$ edges. The procedure
\proc{Faster-All-Pairs-Shortest-Paths} can therefore be modified as follows:

\begin{codebox}
\Procname{$\proc{Contains-Negative-Cycle}(W)$}
\li $n = W.rows$
\li $L^{(1)} = W$
\li $m = 1$
\li \While $m < n - 1$ \Do
\li   $L^{(2m)} = \proc{Extend-Shortest-Paths}(L^{(m)}, L^{(m)})$
\li   \For $i = 1 \To n$ \Do
\li     \If $l_{ii}^{(2m)} < 0$ \Then
\li       \Return true \End \End \End
\li \Return false
\end{codebox}

The running time is dominated by \proc{Extend-Shortest-Paths} and therefore
the same as in original, that is $\Theta(n^3 \lg n)$.

\end{solution}

\end{document}
